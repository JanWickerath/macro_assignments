
\section*{Problem 1}

\paragraph{1.)}
Use the Arrow-Debreu equilibrium structure (see Krueger chapter 8) which
assumes that all trading takes place in a fictional period 0 and no trade
afterwards. Then households maximize their lifetime utility subject to their
lifetime budgetconstraint
\begin{align*}
  \max_{c_t^y, c_{ŧ+1}^o} u\left(c_t^y\right) + u\left(c_{t+1}^o\right) \\
  \text{s. t.}~~p_t c_t^y + p_{t+1} c_{t+1}^o = p_t e^y + p_{t+1} e^o
\end{align*}
Old households maximize their period consumption subject to their endowment
\begin{align*}
  max_{c_1^o} u\left(c_1^o\right) \\
  \text{s.t.}~~p_1 c_1^o = p_1 e^o + m
\end{align*}
additionally goods markets have to clear
\begin{equation*}
  c_t^y + c_t^o = e_y + e_o
\end{equation*}

Assuming log utility we get from the first order condition of the household
that 
\begin{equation*}
  \frac{1}{c_t^y} = \frac{p_t}{p_{t+1}} \frac{1}{c_{t+1}^o} \iff
  \frac{p_{t+1}}{p_t} = \frac{c_t^y}{c_{t+1}^o}
\end{equation*}

Assuming $m = 0$ we immediately get that $c_1^o = e^o$. Plugging this into the
goods market clearing condition delivers $c_1^y = e^y$. Now, using the budget
condition of the household born in period one, we get that 
\begin{align*}
  p_1 \left(c_1^y - e^y\right) &= p_2 \left(e^o - c_2^o\right) \\
  \iff c_2^o &= e^o
\end{align*}
Applying the reasoning iteratively we see that in every period $c_t^y = e^y$,
and $c_t^o = e^o$. To characterize the equilibrium it remains to find an
equilibrium price vector. To find it we use the households first order
condition and plug in the consumption quantities found above, so that
\begin{equation*}
  \frac{p_{t+1}}{p_t} = \frac{e^y}{e^o}
\end{equation*}
So by fixing an initial price value $p_1$ we can get the entire sequence of
prices for the economy.
Finally calculate excess demands:
\begin{align*}
  y = c_t^y - e^y = 0~\forall t\\
  z = c_t^o - e^o = 0~\forall t
\end{align*}

\paragraph{2.)}
Assume the same equilibrium structure as above but no assume $m > 0$. The
representative households FOC looks the same as above and can be reformulated
to
\begin{align*}
  c_t^y = \frac{p_{t+1}}{p_t} c_{t+1}^o \\
  \overset{\text{lifetime budget}}{\implies} ~~ 2 \frac{p_{t+1}}{p_t} c_{t+1}^o =
  \underbrace{e^y}_{\epsilon} + \frac{p_{t+1}}{p_t} \underbrace{e_{t+1}^o}_{1 -
  \epsilon} \\
  \iff ~ c_{t+1}^o = \frac{1}{2} \left[\frac{p_t}{p_{t+1}} \epsilon + \left(1 -
  \epsilon\right) \right] \\
  \implies ~ c_t^y = \frac{1}{2} \left[\epsilon + \frac{p_{t+1}}{p_t} (1 - \epsilon)\right]
\end{align*}
From this we get the excess demands as
\begin{align*}
  y = \frac{1}{2} \left[\frac{p_{t+1}}{p_t} (1 - \epsilon) - \epsilon \right]
  \\
  z = \frac{1}{2} \left[\frac{p_t}{p_{t+1}} \epsilon - (1 - \epsilon) \right]
\end{align*}
Solving $y$ for the price ratio yields
\begin{equation*}
  \frac{p_t}{p_{t+1}} = \frac{1 - \epsilon}{2 y + \epsilon}
\end{equation*}
Plugging this into $z$ and simplifying we get
\begin{equation*}
  z(y) = \frac{\epsilon (1 - \epsilon)}{4 y + 2 \epsilon} - \frac{1 - \epsilon}{2}
\end{equation*}
which is the definition of the offer curve.

\paragraph{3.)}
Note that we know from the first welfare theorem that a sufficient condition
for a competitive equilibrium to be pareto optimal is that
$\sum_{t=1}^{\infty} p_t < \infty$. We have shown before that
$\frac{p_t}{p_{t+1}} = \frac{1 - \epsilon}{2 y + \epsilon}$. If we assume $m =
0$ we can use the result from 1.) that $y = 0$ so that we get
\begin{equation*}
  \frac{p_t}{p_{t+1}} = \frac{1 - \epsilon}{\epsilon} \iff p_{t+1} = p_t
  \frac{\epsilon}{1 - \epsilon} = \left(\frac{\epsilon}{1 - \epsilon} \right)^2
  p_{t-1} = \dots = \left(\frac{\epsilon}{1 - \epsilon}\right)^t p_1
\end{equation*}
To simplify things normalize $p_1 = 1$. Then we get
\begin{align*}
  \sum_{t=1}^\infty p_t = \sum_{t=1}^\infty \left(\frac{\epsilon}{1 -
  \epsilon}\right)^{t-1} = \sum_{t=0}^\infty \left(\frac{\epsilon}{1 - \epsilon}\right)^t
\end{align*}
If we have that $\epsilon < 1/2~\implies~\frac{\epsilon}{1 - \epsilon} < 1$ and
therefore $\sum_{t=1}^\infty p_t < \infty $ from which follows that the no
trade equilibrium in this parameterization is indeed pareto efficient.

\paragraph{4.)}

Reintroducing growth will effect the aggregate resource constraint to be
\begin{equation*}
  c_t^o + (1 + n) c_t^y = e^o + (1 + n) e^y
\end{equation*}
Assume the following taxation scheme: in every period the young have to pay
taxes $\tau$ and the old receive benefits $T$. Assume a sequential equilibrium,
that is the representative household has the following optimization problem
\begin{align*}
  \max_{c_t^y, c_{t+1}^o} \ln(c_t^y) + \beta \ln(c_{t+1}^o) \\
  \text{s.t.}~~c_t^y + s_t^y + \tau &\leq e^y \\
  c_{t+1}^o &\leq e^o + (1 + r_{t+1}) + T
\end{align*}
Additionally in equilibrium the government budget has to be balanced, i.e.
\begin{equation*}
  (1 + n) \tau = T
\end{equation*}
Initial olds budget constraint changes to
\begin{equation*}
  c_1^o \leq e^o + (1 + r_1) m + T
\end{equation*}
The households FOC yields
\begin{equation*}
  c_{t+1}^o = \beta (1 + r_{t+1}) c_t^y
\end{equation*}
We can use the same induction argument as above to see that for $m = 0$ we will
have an optimal household policy of
\begin{equation*}
  s_t^y = 0,~c_t^y = e_y + \tau,~c_{t+1}^o = e^0 + T = e^o + (1 + n) \tau
\end{equation*}
Using this in the households FOC yields
\begin{align*}
  \frac{c_{t+1}^o}{c_t^y} \frac{1}{\beta} = 1 + r_{t+1} \\
  \implies \frac{e^o + (1 + n) \tau}{e^y - \tau} \frac{1}{\beta} &= 1 + r_{t+1}
  \\ 
  &= 1 + r
\end{align*}
Now write households lifetime utility as a function of the tax rate
\begin{equation*}
  V(\tau) = \ln(e_y - \tau) + \beta \ln(e^o + (1 + n) \tau)
\end{equation*}
Assuming that maximizing this with respect to $\tau$ has an interior solution
we get the following first order condition
\begin{align*}
  \beta (1 + n) (e^y - \tau^*) &= e^o + (1 + n) \tau^* \\
  \iff \tau^* &= \frac{\beta}{1 + \beta} e^y - \frac{e^o}{(1 + n)(1 + \beta)} \\
            &= \frac{\beta}{1 + \beta} \epsilon - \frac{1 - \epsilon}{(1 + n)(1 + \beta)}
\end{align*}
Taxation of the young is welfare improving if and only if $\tau^* > 0$.
\begin{align*}
  \implies \beta (1 + n) \epsilon - 1 + \epsilon &> 0 \\
  \iff (1 + n) &> \frac{1 - \epsilon}{\epsilon} \frac{1}{\beta}  \left( = (1 + r)
                 ~\text{if}~\tau = 0 \right)
\end{align*}


%%% Local Variables:
%%% mode: latex
%%% TeX-master: "Assignment2_main"
%%% End:
