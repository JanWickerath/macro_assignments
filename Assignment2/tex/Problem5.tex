
\section*{Problem 5}

\paragraph{1.)}

The assumption $w_y < w_m$ seems natural in a setting where the three periods
are related to a model of the entire lifecycle. In this setting one could argue
that the young invest more in the accumulation of human capital which will
increase their earnings potential in the second mid period of their
lifetime. Writing the model in endowment terms only might be reasonable if one
wants to abstract from the investment decision in human capital. 

\paragraph{2.)}

\begin{equation*}
  \max_{a_t^y, a_{t+1}^m} \ln(w^y - a_t^y) + \beta \ln(w^m + a_t^y - a_{t+1}^m)
  + \beta^2 \ln(a_{t+1}^m)
\end{equation*}
The first order conditions of this problem can be solved to
\begin{align*}
  a_t^y &= w^y - \frac{w^y + w^m}{\beta^2 + \beta + 1} \\
  a_{t+1}^m &= (w^y + w^m) \frac{\beta^2}{\beta^2 + \beta + 1}
\end{align*}

\paragraph{3.)}

So $a^y$ can become negative if and only if
\begin{equation*}
  \underbrace{w^y}_{\text{endowment when young}} < \underbrace{\frac{w^y +
      w^m}{\beta^2 + \beta + 1}}_{\text{discounted lifetime endowment}}
\end{equation*}

\textbf{Solution attempts to the extension can be found in the attached jupyter
notebook.}

%%% Local Variables:
%%% mode: latex
%%% TeX-master: "Assignment2_main"
%%% End:
