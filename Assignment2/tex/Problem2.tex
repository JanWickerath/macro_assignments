
\section*{Problem 2}

Now every individual is endowed with one unit of labor when young and nothing
when old. The initial old are endowed with $(1 + n) k_1$ units of capital,
which is storable. Additional to households there exist firms, which employ
labor and capital to produce the consumption good according to the production
function $F(K_t,~L_t) = Y_t$. The firm pays prices $r_t,~w_t$ on its input
factors. Then a competitive equilibrium is characterized by an initial capital
stock $k_1$, a consumption allocation of households, an allocation of factor
demands of the firms and prices for labor and capital, such that households
\begin{align*}
  \max_{c_t^y, c_{t+1}^o, s_t^y}  u(c_t^y) + \beta u(c_{t+1}^o) \\
  \text{s.t.}~~c_t^y + s_t^y &\leq w_t \\
  c_{t+1}^o &\leq (1 + r_{t+1} - \delta) s_t^y
\end{align*}
where $\delta$ is the depreciation rate of capital. The initial old maximize
their period utility subject to
\begin{equation*}
  c_1^o \leq (1 + r_1 - \delta) k_1
\end{equation*}

Given input prices firms maximize profits:
\begin{equation*}
  \max_{K_t,L_t} F(K_t,~L_t) - r_t K_t - w_t L_t
\end{equation*}
and all markets (goods, capital, labor) clear
\begin{align*}
  N_t^y c_t^y + N_t^o c_t^o + K_{t+1} - (1 - \delta) K_t &= F(K_t,~L_t) \\
  N_t^y s_t^y &= K_{t+1} \\
  N_t^y &= L_t
\end{align*}

Solving the households problem leads to the following first order condition
\begin{equation*}
  u'(c_t^y) = \beta (1 + r_{t+1} - \delta) u'(c_{t+1}^o)
\end{equation*}

Now assume log utility and a Cobb douglas production function with constant
returns to scale $F(K,~L) = K^\alpha L^{1 - \alpha}$. To solve for the law of
motion of capital first rewrite the asset market clearing condition
\begin{equation*}
  K_{t+1} = N_t^y s_t^y \iff (1 + n) k_{t+1} = s_t^y 
\end{equation*}
with $k_t = K_t / L_t$. We know that $w_t = \frac{\partial F(K_t,
  L_t)}{\partial L_t} = (1 - \alpha) k_t^\alpha$. Now use that
$c_{t+1}^o = (1 + r_{t+1} - \delta) s_t^y$ and plug into the households FOC,
which under log utility can be written as
\begin{align*}
  & c_{t+1}^o = \beta (1 + r_{t+1} - \delta) c_t^y \\
  &\iff (1 + r_{t+1} - \delta) s_t^y = \beta (1 + r_{t+1} - \delta) c_t^y \\
  &\iff c_t^y = \frac{s_t^y}{\beta} \\
  &\implies \frac{s_t^y}{\beta} + s_t^y = w_t \\
  &\iff s_t^y = w_t \frac{\beta}{1 + \beta} \\
  &\implies (1 + n) k_{t+1} = (1 - \alpha) k_t^\alpha \frac{\beta}{1 + \beta}
\end{align*}
The last equation above defines the law of motion for capital. To finde the
steady state assume that $k_{t+1} = k_t = k$ and solve the equation above for
$k$.
\begin{align*}
  (1 + n) k &= (1 - \alpha) k^\alpha \frac{\beta}{1 + \beta} \\
  \iff k &= \left[\frac{(1 - \alpha) \beta}{(1 + n) (1 +
           \beta)}\right]^{\frac{1}{1 - \alpha}}
\end{align*}

To find the golden rule capital stock (not asked for here, but needed for the
next exercise) find the steady state capital stock that maximizes total period
consumption $c^y + c^o / (1 + n) = (n + \delta) k - f(k)$.
Taking the first derivative with respect to $k$ of the right hand side leads to
the following first order condition:
\begin{align*}
  (n + \delta) - f'\left(k^{GR}\right) = 0 \\
  \iff n + \delta = f'\left(k^{GR}\right)
\end{align*}
In the cobb douglas case $f'(k) = \alpha k^{\alpha - 1}$ so that
\begin{align*}
  n + \delta = \alpha k_{GR}^{\alpha - 1} \\
  \iff k^{GR} = \left(\frac{n + \delta}{\alpha}\right)^{\frac{1}{\alpha - 1}}
\end{align*}


\textbf{For Problems 3 and 4 see attached Jupyter notebooks}


%%% Local Variables:
%%% mode: latex
%%% TeX-master: "Assignment2_main"
%%% End:
