
\section*{Problem 2: McCall Job Search model}
\label{problem2}

\paragraph{a.}

A Worker has period utility $u(y) = y$, where $y$ is the period income. He can
either be employed at wage $w$ or unemployed in which case he receives
unemployment benefits $b$. If he is employed he receives $w$ forever,
i.e. he has no risk of losing his job and cannot search for a new job
while employed. If he is unemployed he receives $b$ today and is offered a job
with wage $w$ from the nex period onward. The wage offer $w$ is a random draw
from some known probability distribution.

Given the structure described above the households indirect lifetime utility
when employed at wage $w$ can be written recursively as
\begin{equation}
  \label{eq:value_employed}
  V_e(w) = u(w) + \beta V_e(w)
\end{equation}
and when unemployed as
\begin{equation}
  \label{eq:value_unemp}
  V_u = u(b) + \beta \mathbb{E} \left[\max \{V_u, V_e(w)\} \right]
\end{equation}
Note that we can rewrite equation \eqref{eq:value_employed} as 
\begin{align}
\label{eq:V_e}
  V_e(w) (1 - \beta) &= u(w) \nonumber\\ 
  V_e(w) &= \frac{u(w)}{1 - \beta}
\end{align}
Now we can combine \eqref{eq:value_unemp} and \eqref{eq:V_e} to recursively
write the value of a job offer in period $t$ $J_t(w)$ as
\begin{equation}
\label{eq:bellman}
  J_{t}(w) = \max \left\{ \frac{u(w)}{1 - \beta}, u(b) + \beta \mathbb{E}
    \left[J_{t+1}(w)\right] \right\}
\end{equation}
Obviously the optimal solution for the worker is to accept all offers for which
it holds that 
\begin{equation*}
  V_e(w) \geq V_u = u(b) + \beta \mathbb{E} \left[J(w)\right]
\end{equation*}
where $J(w)$ is the functional fixed point in equation \eqref{eq:bellman}. If
it holds with with equality it implicitly defines the wage level $\bar{w}$ at
which the worker is indifferent between accepting the offer and rejecting
it. This wage is called ``reservation wage''. So for the specified utility
function it follows that
\begin{equation}
\label{eq:res_wage}
  \frac{\bar{w}}{1 - \beta} = b + \beta \mathbb{E} \left[J(w)\right]
\end{equation}
From the definition of the wage offer it follows immediately that
\begin{align*}
  J(w) &= 
  \begin{cases}
    \frac{w}{1 - \beta}~ \text{if}~w > \bar{w} \\
    \frac{\bar{w}}{1 - \beta}~\text{if}~w \leq \bar{w}
  \end{cases}\\
  \implies  \mathbb{E} \left[J(w)\right] &= \mathbb{E} \left[ \max \left\{\frac{w}{1 -
  \beta}, \frac{\bar{w}}{1 - \beta}  \right\}\right] \\
  &= \frac{1}{1 - \beta} \mathbb{E} \left[ \max \left\{w, \bar{w} \right\} \right]
\end{align*}
Plugging this into equation \eqref{eq:res_wage} we get that
\begin{align}
\label{eq:wage_bellman}
  \frac{\bar{w}}{1 - \beta} &= b + \frac{\beta}{1 - \beta} \mathbb{E} [\max(w,
                              \bar{w})] \nonumber\\
                    \bar{w} &= (1 - \beta) b + \beta \mathbb{E} [\max (w, \bar{w})]
\end{align}
Note that we can rewrite the expectation operator above as
\begin{align*}
  \mathbb{E} \max[w, \bar{w}] &= \bar{w}~P(w \leq \bar{w}) + \mathbb{E} [w | w >
                                \bar{w}] \\
                              &= \bar{w} \int_0^{\bar{w}} f(w) dw + \int_{\bar{w}}^{w^{\text{Daron}}} w f(w) dw
\end{align*}
We can now find the reservation wage as a fixed point by iteratively using the
right hand side of equation \eqref{eq:wage_bellman} and characterising the
expectation as above. This procedure can be applied because the operator on the
right hand side is a contraction and according to the contraction mapping
theorem it has a unique fixed point that can be reached iteratively from any
arbitrary starting guess for $\bar{w}$.

\paragraph{b.}

\newtheorem{claim}{Claim}
\begin{claim}
  Let $X$ be the interval $(0, w^{\text{Daron}})$ and let
  $J:~X \to X$ be some bounded function. Let $b$ and $\beta$ be
  parameters such that $b \in \mathbb{R}_+$ and $\beta \in (0, 1)$. Then the
  operator $T$ defined by $T(J(w)) = \max \left[ \frac{w}{1 - \beta}, b +
    \beta \mathbb{E} \left(J(w)\right)\right]$ has a unique fixed point in the
  space of bounded functions.
\end{claim}
\begin{proof}
  Denote the space of bounded function by $\mathbb{B(X)}$. To
  establish that $T$ has a unique fixed points we need to show that it is a
  contraction. Once that is established the Contraction mapping theorem
  immediately gives the result. To proof that it is indeed a contraction we can
  refer to Blackwell's sufficient conditions, which state that it suffices to
  show that $T$ fullfills two conditions, monotonicity and discounting. 

  First show monotonicity: Take two arbitrary functions $f$ and $g$ in
  $\mathbb{B}$ such that, without loss of generality, $f(x) \leq g(x)~\forall x
  \in X$. Assume by contradiction that
  \begin{align*}
    (Tf)(w) &> (Tg)(w) \\
    \implies \max \left[\frac{w}{1 - \beta}, b + \beta \mathbb{E}(f(w))\right]
            &> \max \left[\frac{w}{1 - \beta}, b + \beta \mathbb{E}(g(w))\right]
  \end{align*}
  First look at the case where the value of being employed $\frac{w}{1 -
    \beta}$ is lower than being unemployed in both cases. Then the above
  immediately implies that
  \begin{align*}
    b + \beta \mathbb{E}(f(w)) &> b + \beta \mathbb{E}(g(w)) \\
    \iff \mathbb{E}(f(w)) &> \mathbb{E}(g(w))
  \end{align*}
  which contradicts the assumption that $f>g~\forall x$. If instead the the
  value of being employed is larger than being unemployed in both cases. In
  this case it obviously holds that $(Tf)(w) = (Tg)(w)$ which is also a
  contradiction. As it also holds that $\mathbb{E}(f(w)) \leq \mathbb{E}(g(w))$
  it cannot be that the value of being employed is higher then the value of
  being unemployed under the function $g$ but not under $f$. So the assumption
  that $(Tf)(w) > (Tg)(w)$ can never hold and hence the operator $T$ fulfills
  monotonicity.

  Second look whether it also fulfills discounting: Take some arbitrary $a \geq
  0,~w \in X,~f \in \mathbb{B}$.
  \begin{align*}
    \left[T(f + a)\right] (w) &= \max \left[ \frac{w}{1 - \beta}, b + \beta
                                \mathbb{E} (f(w) + a)\right] \\
                              &= \max \left[ \frac{w}{1 - \beta}, b + \beta
                                \mathbb{E} \left[f(w)\right] + \beta a\right]
  \end{align*}
  Now take the following two cases:
  \begin{enumerate}
  \item
    \begin{align*}
      \frac{w}{1 - \beta} > b + \beta \mathbb{E} \left[f(w)\right] + \beta a
      \geq b + \beta \mathbb{E} \left[f(w)\right] \\
      \implies \left[T(f + a)\right] (w) = \frac{w}{1 - \beta} \leq \frac{w}{1
      - \beta} + \beta a = (Tf)(w) + \beta a
    \end{align*}
  \item
    \begin{align*}
      \frac{w}{1 - \beta} \leq b + \beta \mathbb{E} \left[f(w)\right] + \beta a
      \\
      \implies \left[T(f + a)\right](w) = b + \beta \mathbb{E} \left[f(w)\right] + \beta a
    \end{align*}
    \begin{enumerate}
    \item
      \begin{align*}
        \frac{w}{1 - \beta} < b + \beta \mathbb{E} \left[f(w)\right] + \beta a
        \\
        \implies (Tf)(w) + \beta a = b + \beta \mathbb{E} \left[f(w)\right] +
        \beta a = \left[T(f + a)\right](w)
      \end{align*}
    \item
      \begin{align*}
        \frac{w}{1 - \beta} \geq b + \beta \mathbb{E} \left[f(w)\right] \\
        \implies (Tf)(w) + \beta a = \frac{w}{1 - \beta} + \beta a \\
        \geq b + \beta \mathbb{E} \left[f(w)\right] + \beta a \\
        = \left[T(f + a)\right] (w)
      \end{align*}
    \end{enumerate}
  \end{enumerate}
  Hence in all possible cases $T$ fulfills discounting and therefore we know
  that it is a contraction and has a unique fixed point in $\mathbb{B}$ that
  can be reached iteratively from every arbitrary bounded function in
  $\mathbb{B}$. 
\end{proof}

%%% Local Variables:
%%% mode: latex
%%% TeX-master: "Assignment3_main"
%%% End:
