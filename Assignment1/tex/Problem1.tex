\section*{Problem 1}
\paragraph{1.}
Note that you can write $c^{1 - \sigma}$ as $\exp\left(\ln c^{1 - \sigma}\right) = \exp\left(\left[1 - \sigma\right] \ln \left(c \right) \right)$. Hence:
\begin{equation*}
\lim_{\sigma \to 1} \frac{c^{1 - \sigma} - 1}{1 - \sigma} = \lim_{\sigma \to 1} \frac{\exp\left(\left[1 - \sigma\right] \ln \left(c \right) \right)}{1 - \sigma}
\end{equation*}
As numerator and denominator both go to $0$ as $\sigma $ approaches $1$ we can apply L'Hopital's Rule to compute the limit:
\begin{align*}
\lim_{\sigma \to 1} \frac{\exp\left(\ln\left[\left(1 - \sigma\right) c\right]\right)}{1 - \sigma} &= \lim_{\sigma \to 1} \frac{\exp\left[\left(1 - \sigma\right) \ln \left(c \right) \right] \left(-\sigma\right) \ln\left(c\right)}{-1} \\
 &= \lim_{\sigma \to 1} \exp\left[\left(1 - \sigma\right) \ln\left(c\right)\right] \sigma \ln \left(c\right) \\
 &= \ln c
\end{align*}

\paragraph{2.}
The households optimization problem can be written down as:
\begin{align*}
&\max_{c,~h} \frac{c^{1 - \sigma} - 1}{1 - \sigma} - \frac{h^{1 + \chi}}{1 + \chi}, ~~\text{s.t.}~c = w h \left(1 - \tau_l\right) + T \\
\implies &\max_h \frac{\left[w h \left(1 - \tau_l\right) + T\right]^{1 - \sigma} - 1}{1 - \sigma} - \frac{h^{1 + \chi}}{1 + \chi}
\end{align*}
Then the FOC with respect to $h$ is:
\begin{equation*}
\left[w h \left(1 - \tau_l\right) + T\right]^{-\sigma} w \left(1 - \tau_l\right) - h^\chi = 0
\end{equation*}
Now use that $T = w \tau_l h$ in FOC and Budget constraint to get the households optimality conditions:
\begin{align}
c^{-\sigma} w \left(1 - \tau_l\right) - h^\chi &= 0 \\
w h - c &= 0
\end{align}

\begin{figure}[h]
	\centering
			\includegraphics[width=0.99\textwidth]{figures/problem1.pdf}
	\caption{Optimal Policies, Transfers and welfare with varying $\chi $}
	\label{fig:problem1}
\end{figure}

\begin{figure}[h]%
\includegraphics[width=1\columnwidth]{figures/problem1_ext2.pdf}%
\caption{Optimal Policies, Transfers and welfare with varying $\sigma $}%
\label{fig:problem1_ext2}%
\end{figure}

\paragraph{Extensions}
Figures \ref{fig:problem1} and \ref{fig:problem1_ext2} show the optimal
consumption and hours worked decision of the household, the governments
transfer revenues, and the households welfare as a function of the labor tax
rate $\tau_l$. The first figure plots the functions for different values of the
frisch labor supply elasticity $\chi $ and the second for different values of
$\sigma $.

Given the Transfers plotted in figure \ref{fig:problem1} it can not be
plausibly argued that a decline in the labor tax rate is self--financing as tax
revenue peaks somewhere around $0.7$ to $0.9$ for the different values of
$\chi $.

With figure \ref{fig:problem1_ext2} one could argue that a decline in the labor
tax rate is self--financing, if one assumes a very small value of $\sigma
$. Here tax revenue for $\sigma = 0.1$ peaks at around $0.5$ and the current
effective labor tax rate, including all social security contributions, might be
somewhat higher than that.

For the given utility function $\sigma $ is the Arrow-Pratt measure of relative
risk aversion, which can be calculated as $-c \frac{\partial^2 V / \partial
  c^2}{\partial V / \partial c} $

%%% Local Variables:
%%% mode: latex
%%% TeX-master: "Assignment1_main"
%%% End: