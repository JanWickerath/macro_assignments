
\section*{Problem 4}
\label{problem4}


\paragraph{Household Problem}

Assuming log utility the household problem can be written as
\begin{align*}
  \max_{c_i, h_i} \log \left[ c_i - \frac{h_i^{1 + \psi }}{1 + \psi } \right]
  \\
  \text{s.t.} ~ c_i = \left(1 - \tau\right) w_i h_i + T
\end{align*}
Substituting the budget constraint into the utility function and taking the
first derivative with respect to hours worked yields the following first order
condition:
\begin{equation*}
  \frac{1}{c_i - \frac{h_i^{1 + \psi }}{1 + \psi }} \left[ \left(1 - \tau
    \right) w_i - h_i^\psi \right] \overset{!}{=} 0 
\end{equation*}
After rearranging and using the budget constraint we get the following optimal
consumption and labor supply policy of the households:
\begin{align*}
  h_i^* &= \left[ \left(1 - \tau\right) w_i \right]^{\frac{1}{\psi }} \\
  c_i^* &= \left[ \left(1 - \tau\right) w_i \right]^{\frac{1 + \psi }{\psi }} + T
\end{align*}

Using this we can derive the marshallian elasticity as follows:
\begin{align*}
  \frac{\partial h_i}{\partial w_i} \frac{w_i}{h_i} &= \frac{1}{\psi }
  \left[\left(1 - \tau \right) w_i \right]^{\frac{1 - \psi}{\psi }} \frac{w_i
  \left(1 - \tau\right)}{\left[ \left(1 - \tau \right) w_i
  \right]^{\frac{1}{\psi }}} \\
  &= \frac{1}{\psi}
\end{align*}
As we can see the labor supply elasticity is independent of the wage and hence
the wealth of the household, but only depends on the parameter $\psi $.

\paragraph{Social Planers Problem}

Assuming a utilitarian social welfare function
$V^{Gov} = \int_i f\left(w_i\right) U_i d_i $, where $f\left(w_i\right) $
denotes the probability distribution function of wages and $U_i$ the utility of
household $i$, the social planners problem can be written as
\begin{align*}
  \max_{\tau, T} V^{Gov} \\
  \text{s.t.} ~ G + T \leq \int_i \tau w_i h_i d_i \\
  h_i = h_i^* \\
  c_i = c_i^*
\end{align*}
For simplicity assume $G = 0$, i.e. government has no expenditure despite
Transfer payments, and that the budget constraint is binding, i.e. the
government does not waste any resources. Then the lagrangian of the planners
problem is
\begin{align*}
  \mathscr{L} = \int_i f\left(w_i\right) U_i d_i - \lambda \left[ T - \int_i \tau w_i
  h_i d_i \right]
\end{align*}
and the corresponding first order conditions are
\begin{align*}
  \frac{d \mathscr{L}}{d \tau } &= \int_i f\left(w_i\right) \frac{d U_i}{d \tau}
  d_i + \lambda \int_i w_i h_i^* d_i \overset{!}{=} 0 \\
  \frac{d \mathscr{L}}{d T} &= \int_i \left(w_i\right) \frac{d U_i}{d T} d_i -
  \lambda \overset{!}{=} 0
\end{align*}
with
\begin{align*}
  \frac{d U_i}{d \tau} &= \frac{w_i h_i^*}{c_i^* - \frac{h_i^{*^{1 + \psi}}}{1 +
  \psi}} \left[ \frac{1}{\psi} \left[ \left(1 - \tau\right) w_i 
  \right]^{\frac{1 - \psi}{\psi}} - \frac{1 + \psi}{\psi} \right] \\
  \frac{d U_i}{d T} &= \frac{1}{c_i^* - \frac{h_i^{*^{1 + \psi}}}{1 +
  \psi}}
\end{align*}
Note that the optimal household policies are a function of the exogenous
parameters and the exogenous wage rate, so that the foc's above can be plugged
into a root finding routine to finde the planners optimal policy $\tau^{SP}$
and $T^{SP}$.

\paragraph{Optimal policy for differing $\sigma $}

\begin{figure}[h]
  \centering
  \includegraphics[width=.99\linewidth]{figures/distribs.pdf}
  \caption{Distribution of wages, consumption and labor supply}
\end{figure}

\begin{figure}[h]
  \centering
  \includegraphics[width=.99\linewidth]{figures/opt_policy.pdf}
  \caption{Optimal labor income tax and transfers }
  \label{fig:opt-policy}
\end{figure}